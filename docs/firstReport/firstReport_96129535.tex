\documentclass[12pt]{article}

\usepackage{graphicx}
\usepackage[hypertexnames=false,colorlinks=true,breaklinks]{hyperref}
\usepackage{tikz}
\usepackage{verbatim}
\usepackage{natbib}
\usepackage{apalike}
\usepackage{amsmath,amssymb}

\newcommand{\estNumber}{96129535}

\title{First analysis of Emmett' data}
\author{Joaqu\'{i}n Rapela\thanks{j.rapela@ucl.ac.uk}}

\begin{document}

\maketitle

\section{Introduction}

Here I use svGPFA~\citep{dunckerAndSahani18} to explore the low dimensional
structure of Emmett's switching-task recordings.

\section{Methods}

\subsection{Data}

I epoched the recordings from 0.2 seconds before to mouse entered the first
port (port 2) to 0.2 seconds after it exited the last port (port 7).

I only used trial that were shorter than five seconds (257 trials out of 400
trial with correct sequences).

I only used neurons with a firing rater larger than 0.1 spikes/sec (230
out of 400 neurons).

\subsection{svGPFA model}

The svGPFA model estimates $K$ latent variables in the form of Gaussian
processes:

\begin{align*}
    x_k^{(r)}(t)\sim GP(\mu_k(t), \kappa_k(t,t'))\quad k=1,\ldots,K
\end{align*}

\noindent I concatenate the latent variables $x_k^{(r)}(t)$ into a latent vector
$\mathbf{x}^{(r)}(t)$:

\begin{align}
    \mathbf{x}^{(r)}(t)=\left[\begin{array}{c}
        x_1^{(r)}(t)\\
                            \vdots\\
                            x_K^{(r)}(t)
                        \end{array}\right]
    \label{eq:latentsVector}
\end{align}

The latent variables are combined linearly to produced the log pre-intensity,
$\mathbf{h}^{(r)}(t)=[h_1^{(r)}(t),\ldots,h_N^{(r)}(t)]^\intercal$, where
$h_n^{(r)}(t)$ is the log
pre-intensity of neuron $n$ and trial $r$, for a total of $N$ neurons:

\begin{align}
    \mathbf{h}^{(r)}(t)=C\mathbf{x}^{(r)}(t)+\mathbf{d}
    \label{eq:logPreIntensity}
\end{align}

The log pre-intensity of neuron $n$ and trial $r$ is exponentiated to produce the conditional
intensity function of neuron $n$ and trial $r$:

\begin{align}
    \lambda_n^{(r)}(t)=\exp(h_n^{(r)}(t))
    \label{eq:cif}
\end{align}

\noindent which can be interpreted as the instantaneous firing rate of neuron
$n$ and trial $r$ at time $t$.

The conditional intensity function of neuron $n$ in trial $r$ is used to calculate the
probability of spikes $\{t_{i,n}\}_{i=1}^{\Phi(n,r)}$ in a trial $r$ of length $\tau$:

\begin{align*}
    P(\{t_{i,n}\}_{i=1}^{\Phi(n,r)}|\lambda_n^{(r)}(t))=\exp\left(-\int_0^\tau\lambda_n^{(r)}(t)dt\right)\prod_{i=1}^{\Phi(n,r)}\lambda_n^{(r)}(t_{i,n})
\end{align*}

\section{Results}

\begin{comment}

I estimated svGPFA models with $K=1,\ldots,15$ latent variables
(Eq.~\ref{eq:latentsVector}) and squared
exponential kernels. I selected the
model with 10 latents variables as the optimal one, since for this number of
latent variables the achieved lower bound by the models stop increasing
substantially (Figure~\ref{fig:modelSelection}).

\end{comment}

I arbitrarily set to 10 the number of latent variables \citep[$K$ in Eq.~1
of][]{dunckerAndSahani18}) of the svGPFA model.
% Figure~\ref{fig:estimation-lowerBoundVsIterNo}
% and~\ref{fig:estimation-lowerBoundVsElapsedTime} plot to lower bound as a
% function of the iteration number and elapsed time, respectively.
%
Figures~\ref{fig:orthonornmalized-latent0}-\ref{fig:orthonornmalized-latent9}
plot the orthonormalized latents for all trials.
%
Figure~\ref{fig:3DscatterPlotLatents012} shows a scatter plot of
orthonormalized latents 0, 1 and 2.
%
Figures~\ref{fig:if-neuron0}-\ref{fig:if-neuron16} plot 
intensity functions (Eq.~\ref{eq:cif}), for all trials, of different neurons.
%
Figure~\ref{fig:orthonormalized_embedding} plot the left singular vectors of
the matrix $C$ (Eq.~\ref{eq:logPreIntensity}), corresponding to the
orthonormalized latents, and the offset vector $d$.
%
Figure~\ref{fig:kernels_parameters} displays the estimated lenghtscale
parameters of the squared exponential kernels of all latents.

\section{Conclusions}

The intensity functions if Figures~\ref{fig:if-neuron0}-\ref{fig:if-neuron16}
show diverse tunning of single cells. The majority of them fire more when the
mouse enters the reward port 7 (e.g., neurons 1~ (Figure~\ref{fig:if-neuron1}),
2~(Figure~\ref{fig:if-neuron2}), 4~(Figure~\ref{fig:if-neuron4}),
6~(Figure~\ref{fig:if-neuron6}), 7~(Figure~\ref{fig:if-neuron7}),
17~(Figure~\ref{fig:if-neuron17})). Other neurons fire more when the mouse
enters other ports (e.g., neurons 0~(Figure~\ref{fig:if-neuron0}),
15~(Figure~\ref{fig:if-neuron15}), 16~(Figure~\ref{fig:if-neuron16}) fire more
around port 1, neuron 3~(Figure~\ref{fig:if-neuron3})
around por 3, neuron 12~(Figure~\ref{fig:if-neuron12}) around porr 2).

Despite this diversity at the single cell level, the first three
orthonormalized latents show a pattern at the population level
(Figure~\ref{fig:3DscatterPlotLatents012}). For most trials, there appears to
be an extreme point around the time where the mouse pokes in a port.

In addition, in Fig.~\ref{fig:3DscatterPlotLatents012} trials are sorted
systematically, possibly indicating some type of learning, or drift in neural
activity.

\bibliographystyle{apalike}
\bibliography{latentsVariablesModels}

\listoffigures

\begin{figure}
    \begin{center}

        \href{http://www.gatsby.ucl.ac.uk/~rapela/sthita/reports/firstReport/figures/epochedFirst2In_spikes_times_regionsstriatum__neuron07_maxDurationinf.html}{\includegraphics[width=3in]{../../figures/epochedFirst2In_spikes_times_regionsstriatum__neuron07_maxDurationinf.png}}
        \href{http://www.gatsby.ucl.ac.uk/~rapela/sthita/reports/firstReport/figures/epochedFirst2In_spikes_times_regionsstriatum__neuron00_maxDurationinf.html}{\includegraphics[width=3in]{../../figures/epochedFirst2In_spikes_times_regionsstriatum__neuron00_maxDurationinf.png}}
        \href{http://www.gatsby.ucl.ac.uk/~rapela/sthita/reports/firstReport/figures/epochedFirst2In_spikes_times_regionsstriatum__neuron06_maxDurationinf.html}{\includegraphics[width=3in]{../../figures/epochedFirst2In_spikes_times_regionsstriatum__neuron06_maxDurationinf.png}}
        \linebreak

        \caption{Binned spikes and PSTH (red trace) for neurons 7 (top), 0
        (middle) and 6 (bottom) tunned to port 7, 1 and 3, respectively. Click
        on the image to access its interactive version.}
        \label{fig:binnedSpikesAndPSTHsForNeurons7_0_6}

    \end{center}
\end{figure}

\begin{comment}

\begin{figure}
    \begin{center}

        \href{http://www.gatsby.ucl.ac.uk/~rapela/sthita/reports/firstReport/figures/modelSelectionBlock00.html}{\includegraphics[width=6in]{/nfs/ghome/live/rapela/svGPFA/repos/projects/svGPFA_basal_ganglia/code/slurm/figures/modelSelectionBlock00.png}}

        \caption{Model selection. I selected a model with 10
        latent variables, since at this number the lower bound stopped
        increasing substantially. Click on the
        image to access its interactive version.}
        \label{fig:modelSelection}

    \end{center}
\end{figure}

\end{comment}

\begin{figure}
    \begin{center}

        \href{http://www.gatsby.ucl.ac.uk/~rapela/sthita/reports/firstReport/figures/\estNumber_lowerBoundHistVSIterNo.html}{\includegraphics[width=6in]{../../figures/\estNumber_lowerBoundHistVSIterNo.png}}

        \caption{Lower bound vs iteration number. Click on the
        image to access its interactive version.}
        \label{fig:estimation-lowerBoundVsIterNo}

    \end{center}
\end{figure}


\begin{figure}
    \begin{center}

        \href{http://www.gatsby.ucl.ac.uk/~rapela/sthita/reports/firstReport/figures/\estNumber_lowerBoundHistVsElapsedTime.html}{\includegraphics[width=6in]{../..//figures/\estNumber_lowerBoundHistVsElapsedTime.png}}

        \caption{Lower bound vs elapsed time. Click on the
        image to access its interactive version.}
        \label{fig:estimation-lowerBoundVsElapsedTime}

    \end{center}
\end{figure}

\begin{figure}
    \begin{center}

        \href{http://www.gatsby.ucl.ac.uk/~rapela/sthita/reports/firstReport/figures/\estNumber_orthonormalized_estimatedLatent_latent000.html}{\includegraphics[width=5in]{../../figures/\estNumber_orthonormalized_estimatedLatent_latent000.png}}

        \caption{Orthonormalized latent 0. Orange, red, green, blue and black
        markers correspond to ports 2, 1, 6, 3 and 7, respectively. Crosses and
        circles correspond to poke in and out, respectively.}

        \label{fig:orthonornmalized-latent0}

    \end{center}
\end{figure}

\foreach \i in {1,...,9}{
    \begin{figure}
        \begin{center}

            \href{http://www.gatsby.ucl.ac.uk/~rapela/sthita/reports/firstReport/figures/\estNumber_orthonormalized_estimatedLatent_latent00\i.html}{\includegraphics[width=5in]{../..//figures/\estNumber_orthonormalized_estimatedLatent_latent00\i.png}}

            \caption{Orthonormalized latent \i. Same format as in
            Figure~\ref{fig:orthonornmalized-latent0}. Click on the image to
            access its interactive version.}

            \label{fig:orthonornmalized-latent\i}

        \end{center}
    \end{figure}
}


\begin{figure}
    \begin{center}

        \href{http://www.gatsby.ucl.ac.uk/~rapela/sthita/reports/firstReport/figures/\estNumber_orthonormalized_latents0_1_2_.html}{\includegraphics[width=5in]{../../figures/\estNumber_orthonormalized_latents0_1_2_.png}}

        \caption{Scatter plot of orthonormalized latents 0, 1 and
        2. Same format as in Figure~\ref{fig:orthonornmalized-latent0}. Click
        on the image to access its interactive version.}

        \label{fig:3DscatterPlotLatents012}

    \end{center}
\end{figure}


\foreach \i in {0,1,2,3,4,6,7,8,9}{
    \begin{figure}
        \begin{center}

            \href{http://www.gatsby.ucl.ac.uk/~rapela/sthita/reports/firstReport/figures/\estNumber_CIFsOneNeuronAllTrials_neuron00\i.html}{\includegraphics[width=5in]{../../figures/\estNumber_CIFsOneNeuronAllTrials_neuron00\i.png}}

            \caption{Intensity function for neuron $n=\i$
            ($\lambda_n(t)$, Eq.~\ref{eq:cif}). Same format as in
            Figure~\ref{fig:orthonornmalized-latent0}. Click on the image to
            access its interactive version.}

            \label{fig:if-neuron\i}

        \end{center}
    \end{figure}
}

\foreach \i in {10,11,12,13,14,15,16,17}{
    \begin{figure}
        \begin{center}

            \href{http://www.gatsby.ucl.ac.uk/~rapela/sthita/reports/firstReport/figures/\estNumber_CIFsOneNeuronAllTrials_neuron0\i.html}{\includegraphics[width=5in]{../..//figures/\estNumber_CIFsOneNeuronAllTrials_neuron0\i.png}}

            \caption{Intensity function for neuron \i. Same format as in
            Figure~\ref{fig:orthonornmalized-latent0}. Click on the image to
            access its interactive version.}

            \label{fig:if-neuron\i}

        \end{center}
    \end{figure}
}

\begin{figure}
    \begin{center}

        \href{http://www.gatsby.ucl.ac.uk/~rapela/sthita/reports/firstReport/figures/\estNumber_orthonormalized_embedding_params.html}{\includegraphics[width=5in]{../../figures/\estNumber_orthonormalized_embedding_params.png}}

        \caption{Left singular vectors of the matrix $C$ corresponding to the
        orthonormalized latents, and offset vector $\mathbf{d}$,
        (Eq.~\ref{eq:logPreIntensity}). Click on the image to access its
        interactive version.}

        \label{fig:orthonormalized_embedding}

    \end{center}
\end{figure}

\begin{figure}
    \begin{center}

        \href{http://www.gatsby.ucl.ac.uk/~rapela/sthita/reports/firstReport/figures/\estNumber_kernels_params.html}{\includegraphics[width=6in]{../..//figures/\estNumber_kernels_params.png}}

        \caption{Estimated lenghtscale parameters of the squared
        exponential kernels of all latents. Click on the image to access its
        interactive version.}

        \label{fig:kernels_parameters}

    \end{center}
\end{figure}

\end{document}
